% Documentation for biblatex-caspervector
%
% Copyright (c) 2012-2016,2018 Casper Ti. Vector
%
% This work may be distributed and/or modified under the conditions of the
% LaTeX Project Public License, either version 1.3 of this license or (at
% your option) any later version.
% The latest version of this license is in
%   https://www.latex-project.org/lppl.txt
% and version 1.3 or later is part of all distributions of LaTeX version
% 2005/12/01 or later.
%
% This work has the LPPL maintenance status `maintained'.
% The current maintainer of this work is Casper Ti. Vector.
%
% This work consists of the following files:
%   caspervector.tex
%   caspervector-ay.tex

\documentclass[UTF8]{ctexart}
\usepackage[margin = 2cm, centering, includefoot]{geometry}
\usepackage[
	backend = biber, style = caspervector, utf8,
	giveninits = true, sortgiveninits = true
]{biblatex}
\usepackage{fancyhdr, hyperref, enumitem, fancyvrb, hologo}

\hypersetup{colorlinks = true, allcolors = blue}
\pagestyle{fancy}\fancyhf{}\cfoot{\thepage}
\renewcommand{\headrulewidth}{0pt}
\setlist{nolistsep}
\setlength{\hfuzz}{3pt}
\ctexset{linestretch = {\maxdimen}}
\renewcommand{\bibfont}{\small}
\newcommand{\myemph}[1]{\emph{\textcolor{red}{#1}}}

\DeclareBibliographyCategory{cited}
\AtEveryCitekey{\addtocategory{cited}{\thefield{entrykey}}}
\addbibresource{caspervector.bib}

\begin{document}
\title{\textbf{biblatex 参考文献和引用样式:\texttt{caspervector} v0.3.2}}
\author{%
	Casper Ti.\ Vector\thanks{\ %
		\href{mailto:CasperVector@gmail.com}{\texttt{CasperVector@gmail.com}}.%
	}%
}
\date{2018/06/02}
\maketitle

\section{引言}

传统的 \hologo{BibTeX} 引擎存在一些固有的问题:
首先,其样式文件(\verb|bst| 文件)使用后缀式的栈语言编写,
使开发者难以入门和精通,更不便于一般用户进行自定义;
其次,其排序方式很单一,
无法直接实现中文 \hologo{TeX} 用户常常遇到的按汉语拼音排序等需求。
与此相对应,biblatex\supercite{biblatex}/biber\supercite{biber}
是一套新兴的 \hologo{TeX} 参考文献排版引擎,
其样式文件(设定参考文献样式的 \verb|bbx| 文件和设定引用样式的 \verb|cbx| 文件)
使用 \hologo{LaTeX} 编写,便于学习;
同时,其支持根据 locale 进行排序。

为了个人需要,同时也是为了给社区提供一个有益的参考,
本文作者根据实际应用中遇到的常见需求编写了
\verb|caspervector| 这个中西文 biblatex 样式。
其逻辑框架基于国家标准 GB/T 7714--2005\supercite{gbt7714-2005},
但在后者基础上根据个人审美趣味和 biblatex 所能实现的功能而
对参考文献和引用格式进行了较大幅度的修改。
用户如果需要更严格符合国家标准的样式,可以参考
\verb|gb7714-2015|\supercite{blx-gb7714-2015}。

\section{许可协议}

版权所有 \copyright\ 2012--2018 Casper Ti.\ Vector。%
\verb|caspervector| 参考文献和引用样式以
\hologo{LaTeX} Project Public License\footnote{\ %
	\url{https://www.latex-project.org/lppl/}.%
}发布。%
\verb|caspervector| 样式目前由其作者维护。

\section{系统要求和安装方式}
\subsection{系统要求}

\begin{itemize}
	\item biblatex 宏包(2.0 或以上版本):%
		\verb|caspervector| 样式基于 biblatex 宏包。
	\item biber 程序(和 biblatex 相应的版本):
		用 biber 可以方便地实现文献按字母和拼音顺序排序。
	\item 中文环境:%
		\verb|caspervector| 样式虽支持中文,但其本身不提供中文环境。
		用户仍然需要中文环境才能排版出文档。
\end{itemize}

以上要求在\myemph{最新}的\myemph{完全版}
\hologo{TeX} Live 系统中都有完善的支持。

\section{使用简介}
\subsection{样式调用}

用户应当通过以下命令调用 \verb|caspervector| 样式(如果需要使用
作者--年编码制,需要将 \verb|caspervector| 换成 \verb|caspervector-ay|):
\begin{Verbatim}[frame = single]
% “utf8”可能需要改为“gbk”,根据用户使用的字符编码而定。
% “...” 代表其它选项。
\usepackage[
	backend = biber, style = caspervector, utf8, sorting = cenyt, ...
]{biblatex}
\end{Verbatim}
其中 \verb|sorting| 选项用于(全局)指定按哪些字段排序,
除 biblatex 提供的标准选项\supercite{biblatex}外,%
\verb|caspervector| 样式还提供 \verb|cenyt| 和 \verb|ecnyt| 两种排序方案
(\verb|caspervector| 默认设置为 \verb|none| 排序方案,即按引用顺序排序),
表示依次按文献语言
(\textbf{ce}nyt 表示中文文献在前,\textbf{ec}nyt 表示西文文献在前;
文献语言根据 \verb|language| 字段进行区分,详见第 \ref{sec:fields} 部分)、
编作者姓名(\textbf{n}ame)、出版年(\textbf{y}ear)
和标题(\textbf{t}itle)排序\footnote{\ %
	v0.2.x 和更早版本的 \texttt{caspervector}
	样式支持的额外排序方案为 \texttt{centy} 和 \texttt{ecnty};
	为了支持作者--年编码制,之后的版本不再支持这两种方案。%
}。

参考文献数据库
(\verb|.bib| 文件,其格式见第 \ref{sec:fields}、\ref{sec:entries} 部分)
通过 \verb|\addbibresource| 命令导入,%
\myemph{注意不要省略扩展名 \texttt{.bib}}。
例如,本文的参考文献数据库就是通过下述命令导入的:
\begin{Verbatim}[frame = single]
\addbibresource{caspervector.bib}
\end{Verbatim}
用户可以多次使用 \verb|\addbibresource| 命令,
从多个参考文献数据库中导入参考文献。

\subsection{引用命令}

\verb|caspervector| 样式在顺序编码制下支持 biblatex 所提供的引用命令,
其中最常用的是 \verb|\supercite|、\verb|\parencite| 和 \verb|\cite|。
三个命令的用法类似:
\begin{Verbatim}[frame = single]
% 可选参数 prenote 和 postnote 分别用于设定引用记号前、后的注释。
\citecommand[prenote][postnote]{key}
\end{Verbatim}
其中 \verb|\cite| 产生无格式化的引用标记\footnote{\ %
	biblatex 的默认设置是带方括号,
	而 \texttt{caspervector} 样式中出于功能完备性的考虑去掉了方括号。%
},\verb|\parencite| 产生带方括号的引用标记,
而 \verb|\supercite| 产生上标且带方括号的引用标记\footnote{\ %
	biblatex 的默认设置是只上标、不带方括号,
	而 \texttt{caspervector} 样式中根据作者见到的常见需求加上了方括号。%
}。作者--年编码制下引用命令的用法请参考随 \verb|caspervector|
样式附带的 \verb|caspervector-ay.pdf|(可用 \verb|texdoc| 命令查看)。

例如,在本文中,%
\verb|\parencite{gbt7714-2005}| 的输出是“\parencite{gbt7714-2005}”,
而以下代码的输出是“\cite[文献][第 4 页]{gbt7714-2005}”:
\begin{Verbatim}[frame = single]
\cite[文献][第 4 页]{gbt7714-2005}
\end{Verbatim}

\subsection{文献列表}

使用 \verb|\printbibliography| 命令可以在相应位置排版文献列表。
其可(在方括号内)带一些可选参数\supercite{biblatex},
常见的有:
\begin{itemize}
	\item \verb|title = 标题|:
		可以用于指定文献列表的标题(默认为“参考文献”)。
	\item \verb|heading = 标题样式|:
		最常用的是当 \verb|heading| 的值为 \verb|bibintoc| 时,
		可以将参考文献加入目录中;
		当其值为 \verb|bibnumbered| 时,
		参考文献列表参与章节编号(当然也会被自动加入目录中)。
\end{itemize}

例如,以下代码可以将文献列表的标题改为 “文献”,并使文献列表参与章节编号:
\begin{Verbatim}[frame = single]
\printbibliography[title = {文献}, heading = bibnumbered]
\end{Verbatim}

\subsection{编译方法}

一般情况下,依次执行以下代码即可实现正确的排版:
\begin{Verbatim}[frame = single]
# “texfile”是被 TeX 编译的文件名中除去“.tex”的部分。
# “pdflatex”可改为其它 TeX 程序,使用纯 latex 编译时可能还需要运行 dvipdfmx。
pdflatex texfile
biber -l zh__pinyin texfile
pdflatex texfile
pdflatex texfile
\end{Verbatim}

上述执行 \verb|biber| 的一行命令中,%
\verb|-l| 的参数 \verb|zh__pinyin| 可改为其它
被 Perl 的 \verb|Unicode::Collate| 模块支持的 locale\footnote{\ %
	\url{http://search.cpan.org/~sadahiro/Unicode-Collate/Collate/Locale.pm}.%
},这样在排序时将使用相应的排序规则。
例如,如果要按笔画排序的话,可以将 \verb|zh__pinyin| 改为 \verb|zh__stroke|。

\section{字段介绍}\label{sec:fields}
\subsection{基本字段}

除非特别指出,此部分字段在所有类型条目中均可用。

\begin{itemize}
	\item \verb|author|、\verb|editor|、\verb|translator|:
		作者、编者、译者。\\\myemph{%
			注:
			在析出文献条目中,%
			\texttt{author}、\texttt{editor}、\texttt{translator}
			专指析出文献的作者、编者、译者。
			在 \texttt{@patent} 类条目中,%
			\texttt{author} 也可指专利的持有者。%
		}
	\item \verb|bookauthor|、\verb|booktitle|:析出文献所出自文献的作者和题名。
	\item \verb|title|:文献题名。
	\item \verb|type|:文献类型和电子文献载体标志代码\supercite{gbt7714-2005}。
	\item \verb|location|:出版地,或(在 \verb|@patent| 类条目中)专利申请地。
	\item \verb|publisher|:出版者,或学位论文作者申请学位的单位。
	\item \verb|journal|/\verb|journaltitle|:连续出版物题名,两个字段互相等价。
	\item \verb|year|/\verb|date|:出版年、日期,这两个字段只需填写一个即可。
	\item \verb|volume|:期刊中文献所处的卷号。
	\item \verb|number|:期刊中文献所处的期号,或专利的申请号。
	\item \verb|pages|:文献页码。
	\item \verb|url|/\verb|eid|:
		文献的 URL 和(arXiv 等平台的)电子标识码。\\\myemph{%
			注:%
			\texttt{url} 字段会被自动排版成链接,
			但 \texttt{eid} 字段需要用户手工定义格式;
			后者的一个示例见第 \ref{sec:faq} 部分。%
		}
	\item \verb|urldate|:检索日期,或 URL 的访问日期。
	\item \verb|addendum|:补充说明,排版在文献列表中相应条目的最后。
\end{itemize}

\subsection{特殊字段}

\begin{itemize}
	\item \verb|language|:
		值为 \verb|chinese| 时,相应条目在文献列表中用中文排版;
		否则(为其他值或未定义时)用西文排版。
	\item \verb|note|:用于排版一些特殊内容,参考第 \ref{sec:faq} 部分。
	\item \verb|userf|:\verb|caspervector| 样式内部使用。
	\item 其它通用特殊字段,见 biblatex 手册\supercite{biblatex}。
\end{itemize}

\section{条目类型}\label{sec:entries}
\subsection{\texttt{@book} 类型}

\verb|@book| 类型对应于 GB/T 7714--2005 中所指的“专著”和“电子文献”,
其支持的常见别名包括 \verb|@booklet|、\verb|@online|、\verb|@proceedings|、%
\verb|@report|、\verb|@thesis|、\verb|@unpublished|。

\verb|@book| 类条目必需的基本字段为 \verb|title|。

除必需字段之外,\verb|@book| 类条目也支持以下基本字段:%
\verb|author|、\verb|editor|、\verb|translator|、%
\verb|type|、\verb|location|、\verb|publisher|、%
\verb|year|/\verb|date|(作者--年编码制中必需)、\verb|pages|、%
\verb|url|/\verb|eid|、\verb|urldate|、\verb|addendum|。

\subsection{\texttt{@incollection} 类型}

\verb|@incollection| 类型对应于 GB/T 7714--2005 中所指的“专著中的析出文献”,
其支持的常见别名包括
\verb|@bookinbook|、\verb|@conference|、\verb|@inbook|、\verb|@inproceedings|。

\verb|@incollection| 类条目必需的基本字段为 \verb|title| 以及 \verb|booktitle|。

除必需字段之外,\verb|@incollection| 类条目也支持以下基本字段:%
\verb|author|、\verb|editor|、\verb|translator|、\verb|bookauthor|、%
\verb|type|、\verb|location|、\verb|publisher|、%
\verb|year|/\verb|date|(作者--年编码制中必需)、\verb|pages|、%
\verb|url|/\verb|eid|、\verb|urldate|、\verb|addendum|。

\subsection{\texttt{@periodical} 类型}

\verb|@periodical| 类型对应于 GB/T 7714--2005 中所指的“连续出版物”。

\verb|@periodical| 类条目必需的基本字段为
\verb|title|/\verb|journal|/\verb|journaltitle| 三者中的至少一个。

除必需字段之外,\verb|@periodical| 类条目也支持以下基本字段:%
\verb|author|/\verb|editor|/\verb|translator|、%
\verb|type|、\verb|location|、\verb|publisher|、%
\verb|year|/\verb|date|(作者--年编码制中必需)、%
\verb|volume|、\verb|number|、\verb|pages|、%
\verb|url|/\verb|eid|、\verb|urldate|、\verb|addendum|。

\subsection{\texttt{@article} 类型}

\verb|@article| 类型对应于 GB/T 7714--2005 中所指的“连续出版物中的析出文献”。

\verb|@article| 类条目必需的基本字段为
\verb|journal|/\verb|journaltitle| 两者中的至少一个,
以及 \verb|year|/\verb|date| 两者中的至少一个。

除必需字段之外,\verb|@article| 类条目也支持以下基本字段:%
\verb|author|、\verb|title|、\verb|type|、%
\verb|volume|、\verb|number|、\verb|pages|、%
\verb|url|/\verb|eid|、\verb|urldate|、\verb|addendum|。

\subsection{\texttt{@patent} 类型}

\verb|@patent| 类型用于专利文献。

\verb|@patent| 类条目必需的基本字段为 \verb|title|,
以及 \verb|year|/\verb|date| 两者中的至少一个。

除必需字段之外,\verb|@article| 类条目也支持以下基本字段:%
\verb|author|、\verb|title|、\verb|type|、%
\verb|location|、\verb|number|、%
\verb|url|/\verb|eid|、\verb|urldate|、\verb|addendum|。

\subsection{\texttt{@customf} 类型}

\verb|@customf| 类型为特殊类型,
专用于在文献列表的相应条目中排版自定义的文字。
此类条目必需且唯一支持的基本字段为 \verb|addendum|,
用户可将其设为自己希望排版的内容。

\myemph{%
	注:%
	\texttt{@customf} 类型虽不支持 \texttt{author} 等字段,
	但用户仍可以设定它们的值。
	这样虽不能自动根据这些字段排版,
	但仍可以根据它们
	(主要是 \texttt{language}、\texttt{author}、\texttt{title}
	和 \texttt{year}/\texttt{date} 五个字段)
	进行排序。 %
}

\section{对参考文献进行分类排序}\label{sec:catsort}

使用 biblatex 3.4 或更新版本的用户可以通过对不同的 \verb|\printbibliography|
命令采用不同的调用环境来实现对不同部分文献按不同方案排序。
例如,如需对被引用的文献按照引用顺序排序,
而对未引用的文献按照西文文献在前、中文文献在后排序,
则可以在导言区中加入下列几行代码:
\begin{Verbatim}[frame = single]
% 新建条目分类(category)用于区分被引用和未引用的文献条目。
\DeclareBibliographyCategory{cited}
% 每执行一次除 \nocite 之外的 \cite 类命令,将被引用的文献加到“cited”分类中。
\AtEveryCitekey{\addtocategory{cited}{\thefield{entrykey}}}
\end{Verbatim}
在正文中准备排版文献列表的位置使用如下代码:
\begin{Verbatim}[frame = single]
% 按引用顺序排版“cited”分类,即被引用的文献条目。
\begin{refcontext}[sorting = none]
\printbibliography[category = cited, ..., title = {References}]
\end{refcontext}
% 按西文文献在前、中文文献在后排版“cited”分类之外,即未被引用的文献条目。
\begin{refcontext}[sorting = ecnyt]
\printbibliography[notcategory = cited, ..., title = {Works Not Cited}]
\end{refcontext}
\end{Verbatim}
并在最后一个除 \verb|\nocite| 之外的 \verb|\cite| 类命令之后、%
\verb|\end{document}| 之前的任意合适位置\footnote{\ %
	因为 biblatex 中的引用顺序记录是按每个条目被第一次引用的顺序计算的,
	所以 \texttt{\string\nocite\{*\}} 时导入文献的顺序会覆盖掉后面
	\texttt{\string\cite} 类命令的引用顺序。
}(例如,在本说明文档中,就是在 \verb|\end{document}| 之前一行)
加入以下代码:
\begin{Verbatim}[frame = single]
% 将 .bib 文件中所有的参考文献都加到引用列表中,但不将它们加到“cited”分类中,
% 也不会排版引用标号,只是在最后的 \printbibliography 命令中排版相应的文献条目。
\nocite{*}
\end{Verbatim}

使用 biblatex 2.x 或更旧版本的用户需要
将在正文中准备排版文献列表的位置使用的代码改为:
\begin{Verbatim}[frame = single]
% 按引用顺序排版“cited”分类,即被引用的文献条目。
\printbibliography[category = cited, ..., sorting = none, title = {References}]
% 按西文文献在前、中文文献在后排版“cited”分类之外,即未被引用的文献条目。
\printbibliography%
	[notcategory = cited, ..., sorting = ecnyt, title = {Works Not Cited}]
\end{Verbatim}
biblatex 3.0--3.3 中有一个 bug\footnote{\ %
	可以参考 \url{https://tex.stackexchange.com/questions/250548/}。%
} 导致分类排序失效,此问题基本无解。

\section{FAQ 和其它使用提示}\label{sec:faq}

用户可以通过省略可选字段的方式来避免排版相应的内容。
例如,省略 \verb|type| 字段便可使相应条目不排版文献类型和电子文献载体标志代码。%
\verb|caspervector| 样式不支持 \verb|edition| 字段,
用户可以在 \verb|title| 等字段中手动标注。

biblatex 的标准格式支持通过 \verb|eprint|、\verb|eprintclass|、\verb|eprinttype|
等字段排版 arXiv 等平台的电子标识码;但本文作者因认为其用法死板、实现繁琐,
故使用了自由格式的 \verb|eid| 字段。用户可以使用 \verb|\printtext|
命令调用 biblatex 中内置的相应排版功能,例如文献 \parencite{perelman}
中的电子标识码便是通过以下设定得到的:
\begin{Verbatim}[frame = single]
eid = {\printtext[eprint:arxiv]{math/0211159}},
\end{Verbatim}
其中 \verb|arxiv| 可以替换为其它在标准格式中被 \verb|eprinttype|
字段支持的字符串,详见 biblatex 手册\supercite{biblatex};
如果需要排版 DOI,需要将方括号中的整个字符串替换为 \verb|doi|。%
\myemph”等时,需要将相应字符用反斜杠转义为
	“\texttt{\textbackslash\_}”“\texttt{\textbackslash\%}”等。%
}

用户在很多时候可以通过手动调用格式化命令来临时覆盖预设的格式设定,
例如对于一个 \verb|@article| 类型的条目,以下代码将使年卷期号排版为类似于
“\textbf{2018},\textit{14}(\hologo{TeX} 特刊)”的形式:
\begin{Verbatim}[frame = single]
year = {2018},
volume = {14},
number = {\hologo{TeX} 特刊},
\end{Verbatim}

因为 biblatex 内部实现的缘故,\verb|year|/\verb|date|
字段被设为一些特殊内容时可能导致其出错。如果的确有这样的排版需求,
可以使用 \verb|note| 字段,例如文献 \parencite{1-7}
中的出版年便是通过以下设定得到的(\verb|origdate| 字段不会被排版出来,
但会在排序时被 \verb|caspervector| 样式考虑;
其具体用法请参考 biblatex 手册)\footnote{\ %
	v0.2.x 和更早版本的 \texttt{caspervector} 样式支持直接在
	\texttt{year} 字段中设定特殊内容;在之后的版本中,为了支持作者--年编码制,
	\texttt{caspervector} 的内部实现进行了调整,导致其不再支持这样的用法。%
}:
\begin{Verbatim}[frame = single]
origdate = {1845},
note = {\textbf{1845}(\emph{清同治四年})},
\end{Verbatim}

\section{存在的问题}

如第 \ref{sec:faq} 部分所述,%
\verb|year|/\verb|date| 字段被设为一些特殊内容时可能导致出错。
此外如第 \ref{sec:catsort} 部分所述,%
biblatex 3.0--3.3 的功能调整导致分类排序失效。%
\verb|caspervector| 样式的作者对此表示遗憾,希望用户能谅解。

因为 biblatex 现有功能的限制,一些需求无法直接实现。例如类似于文献
\parencite{3-2} 中同时有出版起止年和起止期号的情况就无法自动排版,
只能通过用户手动实现。下面两种方式均可实现上述需求
(\verb|origdate| 字段的用法请参考第 \ref{sec:faq} 部分):
\begin{Verbatim}[frame = single]
@periodical{3-2,
	author = {中国图书馆学会},
	title = {图书馆学通讯},
	type = {J},
	origdate = {1957/1990},
	note = {\textbf{1957}(1) -- \textbf{1990}(4)},
	location = {北京},
	publisher = {北京图书馆},
	language = {chinese},
}
\end{Verbatim}
或
\begin{Verbatim}[frame = single]
@customf{3-2,
	author = {中国图书馆学会},
	title = {图书馆学通讯},
	date = {1957/1990},
	addendum = {中国图书馆学会。
		\textit{图书馆学通讯} [J]。
		\textbf{1957}(1) -- \textbf{1990}(4)。
		北京:北京图书馆。},
	language = {chinese},
}
\end{Verbatim}
这两种方法中更加推荐使用前者,因为前者只需手动实现出版年和期号的排版。

\begin{refcontext}[sorting = none]
\printbibliography[category = cited, heading = bibnumbered, title = {本文参考文献}]
\end{refcontext}
\begin{refcontext}[sorting = ecnyt]
\printbibliography[notcategory = cited, heading = bibnumbered, title = {%
	其它参考文献示例
	(引自\texorpdfstring{文献 \parencite{gbt7714-2005}}{ GB/T 7714-2005})%
}]
\end{refcontext}

\section{更新记录}
\VerbatimInput[tabsize = 4, fontsize = {\small}, baselinestretch = 1]{ChangeLog.txt}

\nocite{*}
\end{document}

% vim:ft=tex:ts=2:sw=2
