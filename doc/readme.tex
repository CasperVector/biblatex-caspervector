% vim:ft=tex:ts=2:sw=2
%
% Documentation for biblatex-caspervector
%
% Copyright (c) 2012-2014 Casper Ti. Vector
%
% This work may be distributed and/or modified under the conditions of the
% LaTeX Project Public License, either version 1.3 of this license or (at
% your option) any later version.
% The latest version of this license is in
%   http://www.latex-project.org/lppl.txt
% and version 1.3 or later is part of all distributions of LaTeX version
% 2005/12/01 or later.
%
% This work has the LPPL maintenance status `maintained'.
% The current maintainer of this work is Casper Ti. Vector.
%
% This work consists of the following files:
%   readme.tex

\documentclass[UTF8, fancyhdr, hyperref]{ctexart}
\usepackage[margin = 2cm, centering, includefoot]{geometry}
\usepackage[backend = biber, style = caspervector, utf8]{biblatex}
\usepackage{CJKspace, enumitem, fancyvrb, hologo}

\pagestyle{fancy}\fancyhf{}\cfoot{\thepage}
\renewcommand{\headrulewidth}{0pt}
\setlist{nolistsep}
\hypersetup{allcolors = blue}

\DeclareBibliographyCategory{cited}
\AtEveryCitekey{\addtocategory{cited}{\thefield{entrykey}}}
\addbibresource{readme.bib}

\setlength{\hfuzz}{3pt}
\renewcommand{\bibfont}{\small}
\newcommand{\myemph}[1]{\emph{\textcolor{red}{#1}}}

\begin{document}
\title{\textbf{biblatex 参考文献和引用样式:\texttt{caspervector} v0.2.1}}
\author{%
	Casper Ti.\ Vector\thanks{\ %
		\href{mailto:CasperVector@gmail.com}{\texttt{CasperVector@gmail.com}}.%
	}%
}
\date{2015/04/28}
\maketitle

\section{引言}

传统的 \hologo{BibTeX} 引擎存在一些固有的问题:
首先,其样式文件(\verb|bst| 文件)使用后缀式的栈语言编写,
使开发者难以入门和精通,更不便于一般用户进行自定义;
其次,其排序方式很单一,
无法直接实现中文 \hologo{TeX} 用户常常遇到的按汉语拼音排序等需求。

与此相对应,
biblatex\supercite{biblatex}/biber\supercite{biber}
是一套新兴的 \hologo{TeX} 参考文献排版引擎,
其样式文件(设定参考文献样式的 \verb|bbx| 文件和设定引用样式的 \verb|cbx| 文件)
使用 \hologo{LaTeX} 编写,便于学习;
同时,其支持根据 locale 进行排序。

目前 \hologo{TeX} 社区中尚无人发布为中文用户设计的 biblatex 样式。
为了个人需要,同时也是为了给社区提供一个有益的参考,
本文作者根据实际应用中遇到的常见需求编写了
\verb|caspervector| 这个用于顺序编码制的中英文 biblatex 样式。
其逻辑框架基于
国家标准 GB/T 7714--2005\supercite{gbt7714-2005},
但在其基础上根据个人审美趣味和 biblatex 所能实现的功能而
对参考文献和引用格式进行了较大幅度的修改。

\section{许可协议}

版权所有 \copyright\ 2012--2014 Casper Ti.\ Vector。%
\verb|caspervector| 参考文献和引用样式以
\hologo{LaTeX} Project Public License\footnote{\ %
	\url{http://www.latex-project.org/lppl/}.%
}发布。%
\verb|caspervector| 样式目前由其作者维护。

\section{系统要求和安装方式}
\subsection{系统要求}

\begin{itemize}
	\item biblatex 宏包(2.0 或以上版本,\myemph{必需}):%
		\verb|caspervector| 样式基于 biblatex 宏包。
	\item 中文环境(\myemph{必需}):%
		\verb|caspervector| 样式虽支持中文,但其本身不提供中文环境。
		用户仍然需要中文环境才能排版出文档。
	\item biber 程序(和 biblatex 相应的版本,\myemph{可选}):
		用 biber 可以方便地实现文献按字母和拼音顺序排序。
\end{itemize}

以上要求在\myemph{最新}的\myemph{完全版}
\hologo{TeX} Live 系统中都有完善的支持。

\section{使用简介}
\subsection{样式调用}

用户应当通过以下命令调用 \verb|caspervector| 样式:
\begin{Verbatim}[frame = single]
% “utf8”可能需要改为“gbk”,根据用户使用的字符编码而定。
% “...” 代表其它选项。
\usepackage[
	backend = biber, style = caspervector, utf8, sorting = centy, ...
]{biblatex}
\end{Verbatim}
其中 \verb|sorting| 选项用于(全局)指定按哪些字段排序,
除 biblatex 提供的标准选项\supercite{biblatex}外,%
\verb|caspervector| 样式还提供 \verb|centy| 和 \verb|ecnty| 两种排序方案
(\verb|caspervector| 默认设置为 \verb|none| 排序方案,即按引用顺序排序),
表示依次按文献语言
(\textbf{ce}nty 表示中文文献在前,\textbf{ec}nty 表示英文文献在前;
文献语言根据 \verb|language| 字段进行区分,详见第 \ref{sec:fields} 部分)、
编著者姓名(\textbf{n}ame)、标题(\textbf{t}itle)
和出版年(\textbf{y}ear)排序。

参考文献数据库
(\verb|.bib| 文件,其格式见第 \ref{sec:fields}、\ref{sec:entries} 部分)
通过 \verb|\addbibresource| 命令导入,%
\myemph{注意不要省略扩展名 \texttt{.bib}}。
例如,本文的参考文献数据库就是通过下述命令导入的:
\begin{Verbatim}[frame = single]
\addbibresource{readme.bib}
\end{Verbatim}
用户可以多次使用 \verb|\addbibresource| 命令,
从多个参考文献数据库中导入参考文献。

\subsection{引用命令}

\verb|caspervector| 样式支持 biblatex 所提供的引用命令,
其中最常用的是 \verb|\supercite|、\verb|\parencite| 和 \verb|\cite|。
三个命令的用法类似:
\begin{Verbatim}[frame = single]
% 可选参数 prenote 和 postnote 分别用于设定引用记号前、后的注释。
\citecommand[prenote][postnote]{key}
\end{Verbatim}
其中 \verb|\cite| 产生无格式化的引用标记\footnote{\ %
	biblatex 的默认设置是带方括号,
	而 \texttt{caspervector} 样式中出于功能完备性的考虑去掉了方括号。%
},\verb|\parencite| 产生带方括号的引用标记,
而 \verb|\supercite| 产生上标且带方括号的引用标记\footnote{\ %
	biblatex 的默认设置是只上标、不带方括号,
	而 \texttt{caspervector} 样式中根据作者见到的常见需求加上了方括号。%
}。

例如,在本文中,%
\verb|\parencite{gbt7714-2005}| 的输出是“\parencite{gbt7714-2005}”,
而
\begin{Verbatim}[frame = single]
\cite[文献][第 4 页]{gbt7714-2005}
\end{Verbatim}
的输出是“\cite[文献][第 4 页]{gbt7714-2005}”。

\subsection{文献列表}

使用 \verb|\printbibliography| 命令可以在相应位置排版文献列表。
其可(在方括号内)带一些可选参数\supercite{biblatex},
常见的有:
\begin{itemize}
	\item \verb|title = 标题|:
		可以用于指定文献列表的标题(默认为“参考文献”)。
	\item \verb|heading = 标题样式|:
		最常用的是当 \verb|heading| 的值为 \verb|bibintoc| 时,
		可以将参考文献加入目录中;
		当其值为 \verb|bibnumbered| 时,
		参考文献列表参与章节编号(当然也会被自动加入目录中)。
	\item \verb|sorting = 排序方案|:
		对这一部分的文献按照指定的方案排序。此设置会覆盖全局设置。\myemph{%
			注:
			只有同时使用 2.x 或之后版本的 biblatex 和相应兼容版本的 biber,
			才能对每个 \texttt{\string\printbibliography} 命令采用不同的排序方案。%
		}
\end{itemize}

例如,用
\begin{Verbatim}[frame = single]
\printbibliography[title = {文献}, heading = bibnumbered]
\end{Verbatim}
可以将文献列表的标题改为 “文献”,
并使文献列表参与章节编号。

\subsection{编译方法}

一般情况下,依次执行
\begin{Verbatim}[frame = single]
# “texfile”是被 TeX 编译的文件名中除去“.tex”的部分。
# “pdflatex”可改为其它 TeX 程序,使用纯 latex 编译时可能还需要运行 dvipdfmx。
pdflatex texfile
biber -l zh__pinyin texfile
pdflatex texfile
pdflatex texfile
\end{Verbatim}
即可实现正确的排版。

上述执行 \verb|biber| 的一行命令中,%
\verb|-l| 的参数 \verb|zh__pinyin| 可改为其它
被 Perl 的 \verb|Unicode::Collate| 模块支持的 locale\footnote{\ %
	\url{http://search.cpan.org/~sadahiro/Unicode-Collate/Collate/Locale.pm}.%
},这样在排序时将使用相应的排序规则。
例如,如果要按笔画排序的话,可以将 \verb|zh__pinyin| 改为 \verb|zh__stroke|。

\section{字段介绍}\label{sec:fields}
\subsection{基本字段}

除非特别指出,此部分字段在所有类型条目中均可用。

\begin{itemize}
	\item \verb|author|、\verb|editor|、\verb|translator|:
		作者、编者、译者。\\
		\myemph{%
			注:
			在析出文献条目中,%
			\texttt{author}、\texttt{editor}、\texttt{translator}
			专指析出文献的作者、编者、译者。
			在 \texttt{@patent} 类条目中,%
			\texttt{author} 也可指专利的持有者。%
		}
	\item \verb|bookauthor|、\verb|booktitle|:析出文献所出自文献的作者和题名。
	\item \verb|title|:文献题名。
	\item \verb|type|:文献类型和电子文献载体标志代码\supercite{gbt7714-2005}。
	\item \verb|location|:出版地,或(在 \verb|@patent| 类条目中)专利申请地。
	\item \verb|publisher|:出版者。
	\item \verb|journal|/\verb|journaltitle|:连续出版物题名,这两个字段是等价的。
	\item \verb|year|/\verb|date|:出版年、日期,这两个字段只需填写一个即可。
	\item \verb|volume|:期刊中文献所处的卷号。
	\item \verb|number|:期刊中文献所处的期号,或专利的申请号。
	\item \verb|pages|:文献页码。
	\item \verb|url|:文献的 URL。
	\item \verb|urldate|:检索日期,或 URL 的访问日期。
	\item \verb|addendum|:补充说明,排版在文献列表中相应条目的最后。
\end{itemize}

\subsection{特殊字段}

\begin{itemize}
	\item \verb|language|:
		值为 \verb|chinese| 时,相应条目在文献列表中用中文排版;
		否则(为其他值或未定义时)用英文排版。
	\item \verb|userf|:\verb|caspervector| 样式内部使用。
	\item 其它通用特殊字段,见 biblatex 手册\supercite{biblatex}。
\end{itemize}

\section{条目类型}\label{sec:entries}
\subsection{\texttt{@book} 类型}

\verb|@book| 类型对应于 GB/T 7714--2005 中所指的“专著”和“电子文献”,
其支持的常见别名包括 \verb|@booklet|、\verb|@online|、\verb|@proceedings|、%
\verb|@report|、\verb|@thesis|、\verb|@unpublished|。

\verb|@book| 类条目必需的基本字段为 \verb|title|。

除必需字段之外,\verb|@book| 类条目也支持以下基本字段:%
\verb|author|、\verb|editor|、\verb|translator|、%
\verb|type|、\verb|location|、\verb|publisher|、%
\verb|year|/\verb|date|、\verb|pages|、%
\verb|url|、\verb|urldate|、\verb|addendum|。

\subsection{\texttt{@incollection} 类型}

\verb|@incollection| 类型对应于 GB/T 7714--2005 中所指的“专著中的析出文献”,
其支持的常见别名包括
\verb|@bookinbook|、\verb|@conference|、\verb|@inbook|、\verb|@inproceedings|。

\verb|@incollection| 类条目必需的基本字段为 \verb|title| 以及 \verb|booktitle|。

除必需字段之外,\verb|@incollection| 类条目也支持以下基本字段:%
\verb|author|、\verb|editor|、\verb|translator|、\verb|bookauthor|、%
\verb|type|、\verb|location|、\verb|publisher|、%
\verb|year|/\verb|date|、\verb|pages|、%
\verb|url|、\verb|urldate|、\verb|addendum|。

\subsection{\texttt{@periodical} 类型}

\verb|@periodical| 类型对应于 GB/T 7714--2005 中所指的“连续出版物”。

\verb|@periodical| 类条目必需的基本字段为
\verb|title|/\verb|journal|/\verb|journaltitle| 三者中的至少一个。

除必需字段之外,\verb|@periodical| 类条目也支持以下基本字段:%
\verb|author|/\verb|editor|/\verb|translator|、%
\verb|type|、\verb|location|、\verb|publisher|、%
\verb|year|/\verb|date|、\verb|volume|、\verb|number|、\verb|pages|、%
\verb|url|、\verb|urldate|、\verb|addendum|。

\subsection{\texttt{@article} 类型}

\verb|@article| 类型对应于 GB/T 7714--2005 中所指的“连续出版物中的析出文献”。

\verb|@article| 类条目必需的基本字段为
\verb|journal|/\verb|journaltitle| 两者中的至少一个,
以及 \verb|year|/\verb|date| 两者中的至少一个。

除必需字段之外,\verb|@article| 类条目也支持以下基本字段:%
\verb|author|、\verb|title|、\verb|type|、%
\verb|volume|、\verb|number|、\verb|pages|、%
\verb|url|、\verb|urldate|、\verb|addendum|。

\subsection{\texttt{@patent} 类型}

\verb|@patent| 类型用于专利文献。

\verb|@patent| 类条目必需的基本字段为 \verb|title|,
以及 \verb|year|/\verb|date| 两者中的至少一个。

除必需字段之外,\verb|@article| 类条目也支持以下基本字段:%
\verb|author|、\verb|title|、\verb|type|、%
\verb|location|、\verb|number|、%
\verb|url|、\verb|urldate|、\verb|addendum|。

\subsection{\texttt{@customf} 类型}

\verb|@customf| 类型为特殊类型,
专用于在文献列表的相应条目中排版自定义的文字。
此类条目必需且唯一支持的基本字段为 \verb|addendum|,
用户可将其设为自己希望排版的内容。

\myemph{%
	注:%
	\texttt{@customf} 类型虽不支持 \texttt{author} 等字段,
	但用户仍可以设定它们的值。
	这样虽不能自动根据这些字段排版,
	但在仍可以根据它们
	(主要是 \texttt{language}、\texttt{author}、\texttt{title}
	和 \texttt{year} 四个字段)
	进行排序。 %
}

\section{对参考文献进行分类排序}

使用 biber 的用户可以通过
对不同的 \verb|\printbibliography| 命令传递不同的 \verb|sorting| 选项来实现
对不同部分文献按不同方案排序。
例如,如需对被引用的文献按照引用顺序排序,
而对未引用的文献按照英文文献在前、中文文献在后排序,
则可以在导言区中加入下列几行代码:
\begin{Verbatim}[frame = single]
% 新建条目分类(category)用于区分被引用和未引用的文献条目。
\DeclareBibliographyCategory{cited}
% 每执行一次除 \nocite 之外的 \cite 类命令,将被引用的文献加到“cited”分类中。
\AtEveryCitekey{\addtocategory{cited}{\thefield{entrykey}}}
\end{Verbatim}
在正文中准备排版文献列表的位置使用如下代码:
\begin{Verbatim}[frame = single]
% 按引用顺序排版“cited”分类,即被引用的文献条目。
\printbibliography[category = cited, ..., sorting = none, title = {References}]
% 按英文文献在前、中文文献在后排版“cited”分类之外,即未被引用的文献条目。
\printbibliography[notcategory = cited, sorting = ecnty, title = {Works Not Cited}]
\end{Verbatim}
并在最后一个除 \verb|\nocite| 之外的 \verb|\cite| 类命令之后、%
\verb|\end{document}| 之前的任意合适位置\footnote{\ %
	因为 biblatex 中的引用顺序记录是按每个条目被第一次引用的顺序计算的,
	所以 \texttt{\string\nocite\{*\}} 时导入文献的顺序会覆盖掉后面
	\texttt{\string\cite} 类命令的引用顺序。
}(例如,在本说明文档中,就是在两个 \verb|\printbibliography| 命令之间)
加入以下代码:
\begin{Verbatim}[frame = single]
% 将 .bib 文件中所有的参考文献都加到引用列表中,但不将它们加到“cited”分类中,
% 也不会排版引用标号,只是在最后的 \printbibliography 命令中排版相应的文献条目。
\nocite{*}
\end{Verbatim}

\section{FAQ 和其它使用提示}

用户可以通过省略可选字段的方式来避免排版相应的内容。
例如,省略 \verb|type| 字段便可使相应条目不排版文献类型和电子文献载体标志代码。

用户可以通过手动调用格式化命令来临时覆盖预设的格式设定,
例如文献 \parencite{1-7} 中的出版年便是通过设定
\begin{Verbatim}[frame = single]
year = {1845\textmd{\emph{(清同治四年)}}}
\end{Verbatim}
得到的。

\verb|caspervector| 样式不支持 \verb|edition| 字段,
用户可以在 \verb|title| 等字段中手动标注。

直接使用 \verb|userf| 字段是否等于 \verb|zh| 或 \verb|cn|
来区分中英文文献的功能在近期仍将被支持,但在将来可能被去掉或修改。

\section{存在的问题}

因为 biblatex 现有功能的限制,一些需求无法直接实现。
例如类似于文献 \parencite{3-2} 中同时有出版起止年和起止期号的情况就无法自动排版,
只能通过用户手动实现。下面两种方式均可实现上述需求:
(\verb|sortyear| 字段的用法请参考 biblatex 手册\supercite{biblatex}):
\begin{Verbatim}[frame = single]
@periodical{3-2,
	author = {中国图书馆学会},
	title = {图书馆学通讯},
	type = {J},
	sortyear = {1957},
	year = {\textmd{\textbf{1957}(1) -- \textbf{1990}(4)}},
	location = {北京},
	publisher = {北京图书馆},
	language = {chinese},
}
\end{Verbatim}
或
\begin{Verbatim}[frame = single]
@customf{3-2,
	author = {中国图书馆学会},
	title = {图书馆学通讯},
	sortyear = {1957},
	addendum = {中国图书馆学会。
		\textit{图书馆学通讯} [J]。
		\textbf{1957}(1) -- \textbf{1990}(4)。
		北京:北京图书馆。},
	language = {chinese},
}
\end{Verbatim}
这两种方法中更加推荐使用前者,因为前者只需手动实现出版年和期号的排版。

\section{更新记录}
\VerbatimInput[tabsize = 4, fontsize = {\small}, baselinestretch = 1]{ChangeLog.txt}

\printbibliography%
	[category = cited, heading = bibnumbered, sorting = none, title = {本文参考文献}]
\nocite{*}
\printbibliography[
	notcategory = cited, heading = bibnumbered, sorting = ecnty,
	title = {%
		其它参考文献示例
		(引自\texorpdfstring{文献 \parencite{gbt7714-2005}}{ GB/T 7714-2005})
	}
]
\end{document}

