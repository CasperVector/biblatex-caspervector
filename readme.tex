\documentclass[UTF8, fancyhdr, hyperref]{ctexart}
\usepackage[margin = 2cm, centering, includefoot]{geometry}
\usepackage[backend = biber, style = caspervector, defernumbers = true]{biblatex}
\usepackage{hologo}
\usepackage{enumitem}

\pagestyle{fancy}\fancyhf{}\cfoot{\thepage}
\renewcommand{\headrulewidth}{0pt}
\setlist{nolistsep}

\DeclareBibliographyCategory{cited}
\AtEveryCitekey{\addtocategory{cited}{\thefield{entrykey}}}
\addbibresource{readme.bib}
\nocite{*}

\begin{document}
\title{\textbf{biblatex 参考文献和引用样式:\texttt{caspervector}}}
\author{%
	Casper Ti.\ Vector\thanks{%
		\href{mailto:CasperVector@gmail.com}{\texttt{CasperVector@gmail.com}}.%
	}%
}
\date{2012/?/?}
\maketitle

\section{引言}

传统的 \hologo{BibTeX} 引擎存在一些固有的问题:
首先,其样式文件(\verb|bst| 文件)使用后缀式的栈语言编写,
使开发者难以入门和精通,更不便于一般用户进行自定义;
其次,其排序方式很单一,
无法直接实现中文 \TeX{} 用户常常遇到的按汉语拼音排序等需求。

与此相对应,
biblatex\supercite{biblatex}/biber\supercite{biber}
是一套新兴的 \TeX{} 参考文献排版引擎,
其样式文件(设定参考文献样式的 \verb|bbx| 文件和设定引用样式的 \verb|cbx| 文件)
使用 \LaTeX{} 编写,便于学习;
同时,其支持根据 locale 进行排序。

目前 \TeX{} 社区中尚无人发布为中文用户设计的 biblatex 样式。
为了个人需要,同时也是为了给社区提供一个有益的参考,
本文作者根据实际应用中遇到的常见需求编写了
\verb|caspervector| 这个 biblatex 样式。
其逻辑框架基于
国家标准 GB/T 7714--2005\supercite{gbt7714-2005},
但在其基础上根据个人审美趣味和 biblatex 所能实现的功能而
对参考文献和引用格式进行了较大幅度的修改。

\section{字段介绍}
\subsection{基本字段}

除非特别指出,此部分字段在所有类型条目中均可用。

\begin{itemize}
	\item \verb|author|、\verb|editor|、\verb|translator|:
		作者、编者、译者。\\
		\emph{%
			注:
			在析出文献条目中,
			\texttt{author}、\texttt{editor}、\texttt{translator}
			专指析出文献的作者、编者、译者。
			在 \texttt{@patent} 类条目中,
			\texttt{author} 也可指专利的持有者。%
		}
	\item \verb|bookauthor|、\verb|booktitle|:析出文献所出自文献的作者和题名。
	\item \verb|title|:文献题名。
	\item \verb|type|:文献类型和电子文献载体标志代码\supercite{gbt7714-2005}。
	\item \verb|location|:出版地,或(在 \verb|@patent| 类条目中)专利申请地。
	\item \verb|publisher|:出版者。
	\item \verb|journal|/\verb|journaltitle|:连续出版物题名,这两个字段是等价的。
	\item \verb|year|/\verb|date|:出版年、日期,这两个字段只需填写一个即可。
	\item \verb|volume|:期刊中文献所处的卷号。
	\item \verb|number|:期刊中文献所处的期号,或专利的申请号。
	\item \verb|pages|:文献页码。
	\item \verb|url|:文献的 URL。
	\item \verb|urldate|:检索日期,或 URL 的访问日期。
	\item \verb|addendum|:补充说明,排版在文献列表中相应条目的最后。
\end{itemize}

\subsection{特殊字段}

\begin{itemize}
	\item \verb|userf|:
		未定义时,相应条目在文献列表中用英文排版,否则用中文排版。\\
		\emph{%
			注:
			不可用 \texttt{language} 字段区分中英文文献,
			因为 biblatex 使用 babel\supercite{babel} 宏包处理此字段,
			但后者不支持中文,所以可能出错。%
		}
	\item 其它通用特殊字段,见 biblatex 手册\supercite{biblatex}。
\end{itemize}

\section{条目类型}
\subsection{\texttt{@book} 类型}

\verb|@book| 类型对应于 GB/T 7714--2005 中所指的“专著”和“电子文献”类型,
其支持的常见别名包括 \verb|@booklet|、\verb|@proceedings|、
\verb|@report|、\verb|@thesis|、\verb|@unpublished|。

\verb|@book| 类条目必需的基本字段为 \verb|title|。

除必需字段之外,\verb|@book| 类条目也支持以下基本字段:
\verb|author|、\verb|editor|、\verb|translator|、
\verb|type|、\verb|location|、\verb|publisher|、
\verb|year|/\verb|date|、\verb|pages|、
\verb|url|、\verb|urldate|、\verb|addendum|。

\subsection{\texttt{@incollection} 类型}

\verb|@incollection| 类型对应于 GB/T 7714--2005 中所指的“专著中的析出文献”,
其支持的常见别名包括
\verb|@bookinbook|、\verb|@conference|、\verb|@inbook|、\verb|@inproceedings|。

\verb|@incollection| 类条目必需的基本字段为 \verb|title| 以及 \verb|booktitle|。

除必需字段之外,\verb|@incollection| 类条目也支持以下基本字段:
\verb|author|、\verb|editor|、\verb|translator|、\verb|bookauthor|、
\verb|type|、\verb|location|、\verb|publisher|、
\verb|year|/\verb|date|、\verb|pages|、
\verb|url|、\verb|urldate|、\verb|addendum|。

\subsection{\texttt{@periodical} 类型}

\verb|@periodical| 类型对应于 GB/T 7714--2005 中所指的“连续出版物”。

\verb|@periodical| 类条目必需的基本字段为
\verb|title|/\verb|journal|/\verb|journaltitle| 三者中的至少一个。

除必需字段之外,\verb|@periodical| 类条目也支持以下基本字段:
\verb|author|/\verb|editor|/\verb|translator| 
\verb|type|、\verb|location|、\verb|publisher|、
\verb|year|/\verb|date|、\verb|volume|、\verb|number|、\verb|pages|、
\verb|url|、\verb|urldate|、\verb|addendum|。

\subsection{\texttt{@article} 类型}

\verb|@article| 类型对应于 GB/T 7714--2005 中所指的“连续出版物”。

\verb|@article| 类条目必需的基本字段为
\verb|journal|/\verb|journaltitle| 两者中的至少一个,
以及 \verb|year|/\verb|date| 两者中的至少一个。

除必需字段之外,\verb|@article| 类条目也支持以下基本字段:
\verb|author|、\verb|title|、\verb|type|、
\verb|volume|、\verb|number|、\verb|pages|、
\verb|url|、\verb|urldate|、\verb|addendum|。

\subsection{\texttt{@patent} 类型}

\verb|@patent| 类型用于专利文献。

\verb|@patent| 类条目必需的基本字段为 \verb|title|,
以及 \verb|year|/\verb|date| 两者中的至少一个。

除必需字段之外,\verb|@article| 类条目也支持以下基本字段:
\verb|author|、\verb|title|、\verb|type|、
\verb|location|、\verb|number|、
\verb|url|、\verb|urldate|、\verb|addendum|。

\subsection{\texttt{@customf} 类型}

\verb|@patent| 类型为特殊类型,
专用于在文献列表的相应条目中排版自定义的文字。
此类条目必需且唯一支持的基本字段为 \verb|addendum|,
用户可将其设为自己希望排版的内容。

\printbibliography%
	[heading = bibnumbered, title = {本文参考文献}, category = cited]
\printbibliography%
	[heading = bibnumbered, title = {其它参考文献示例}, notcategory = cited]
\end{document}

